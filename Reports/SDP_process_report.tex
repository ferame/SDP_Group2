\documentclass[12pt,oneside]{article}
\usepackage[margin=1in, a4paper]{geometry}
\usepackage{amsmath}
\usepackage[T1]{fontenc}
\usepackage{enumerate}
\usepackage{graphicx}
\usepackage[export]{adjustbox} % loads also graphicx
\usepackage{wrapfig}
\usepackage{epstopdf}
\usepackage{multicol}
\usepackage{amsmath}
\usepackage{enumitem}
\usepackage{lipsum}  % generates filler text
\usepackage{listings}
\usepackage{color}
\usepackage{hyperref}
\begin{document}
	\begin{titlepage}
		
		\newcommand{\HRule}{\rule{\linewidth}{0.5mm}} % Defines a new command for the horizontal lines, change thickness here
		
		\center % Center everything on the page
		
		%----------------------------------------------------------------------------------------
		%	HEADING SECTIONS
		%----------------------------------------------------------------------------------------
		
		\textsc{\LARGE Software Development Project}\\[1.5cm]
		\textsc{\Large Team A}\\[0.5cm] % Minor heading such as course title
		\textsc{\Large Group 2}\\[0.5cm] % Major heading such as course name	
		%----------------------------------------------------------------------------------------
		%	TITLE SECTION
		%----------------------------------------------------------------------------------------
		
		\HRule \\[0.4cm]
		{ \huge \bfseries Process Report}\\[0.4cm] % Title of your document
		\HRule \\[1.5cm]
		
		%----------------------------------------------------------------------------------------
		%	AUTHOR SECTION
		%----------------------------------------------------------------------------------------
		
		\begin{minipage}[t]{0.4\textwidth}
			\begin{flushleft} \large
				\emph{Team:} \\
				Stefan \textsc{Ivanov}\\
				Oana \textsc{Radu}\\
				Miro \textsc{Trifonov}\\
				Asta \textsc{Bogdelyte}\\
				Jianmeng \textsc{Yu}\\
				Justin \textsc{Alisauskas}\\
				Rosen \textsc{Chakarov}\\
			\end{flushleft}
		\end{minipage}
		~
		\begin{minipage}[t]{0.4\textwidth}
			\begin{flushright} \large
				\emph{Supervisor:} \\
				Angus \textsc{Pearson}
			\end{flushright}
		\end{minipage}\\[2cm]
		
		% If you don't want a supervisor, uncomment the two lines below and remove the section above
		%\Large \emph{Author:}\\
		%John \textsc{Smith}\\[3cm] % Your name
		
		%----------------------------------------------------------------------------------------
		
		\vfill % Fill the rest of the page with whitespace
		%----------------------------------------------------------------------------------------
		%	LOGO SECTION
		%----------------------------------------------------------------------------------------
		
		\includegraphics[width=50mm,scale=0.5]{unilogo.jpg}\\[1cm] % Include a department/university logo - this will require the graphicx package
		
		%----------------------------------------------------------------------------------------
		
		\vfill % Fill the rest of the page with whitespace
		%----------------------------------------------------------------------------------------
		%	DATE SECTION
		%----------------------------------------------------------------------------------------
		
		{\large \today}\\[2cm] % Date, change the \today to a set date if you want to be precise
	
		
	\end{titlepage}
	\section{Scope and purpose}
		The following process report is made to detail our interactions within the group and team based on a comprehensive strategy of project development in order to achieve our target of improving both the hardware and software of the Fred robot.
	\section{Communication}
		Our first project management decision was regarding the manner of communicating among all the individuals involved in the project. As we wanted from the beginning to adhere to the agile methodology, this step was paramount for our further interactions.
	\subsection{Group communication}
		We decided to use Slack as our main communication option. We set up our own Slack group, as we needed the admin privileges to set up GitHub plugin for tracking of commits. We also added separate channels for hardware, software, report and work scheduling.
		
		An important aspect of our communication were the regular meetings we held. Each week we have an informal meeting with all the group members and a formal meeting with our mentor to discuss progress, identify the key areas which we have problems with and set the agenda for moving forward. In addition to this, we have shorter (10-15 min) technical meetings every day, during which we give updates on what each of us changed since the last time and set our daily goals. Also, we would share this information on Slack to keep everyone updated at all times and we would consult each other on any changes that we have made to the code or the robot itself.
	\subsection{Team communication}
		An additional Slack group was created for discussions between Group 1 and Group 2, as we thought that it will be beneficial to keep each other updated on the progress that our groups make. However, we have not yet worked together apart from giving each other general advice, as we decided to wait for the team phase to begin so that we can solely concentrate on individual performance until then.
		
	\section{Task allocation}
	\subsection{Initial stage}
		From the beginning we divided the group members into hardware and software teams- each subteam participated in the according workshops( Justin, Jianmeng and Rosen at the hardware workshop and Stefan, Asta, Oana at the software workshop). From the information gathered from these workshops, we knew that our main first task needed to be sorting the robot design itself. As we did not understand the code well enough, we split everyone into software and hardware teams, without allocating the software team to smaller subteams (Vision, Strategy and Communication) as we deemed it to be a lot more sensible to split the tasks when people are familiar with the code and can state which part they would prefer to work on. For the first couple of days the software team (Stefan, Miro, Asta, Oana) had a task of analysing several Git repositories of robots code from previous year, so that we could decide which code will be used as a base for our own implementation. The hardware team (Jianmeng and Justin) had a task of building a basic robot for an initial testing of the system.
	\subsection{Further development}
		At the start of the second week our software team had already familiarised themselves with the code and the hardware team managed to develop several prototypes; one of the prototypes was a copy of Fred which was used as a testbed for arduino code and another prototype was a first iteration of design for a robot with three holonomic wheels.
		\newline
		\emph{robots photooooos with text}
		\newline
		Afterwards, we decided to make improvements to our group structure. The software team was split into subteams: Vision - Asta, Strategy - Stefan, Miro, Rosen, Oana, Communication - Miro. The hardware team remained the same - Jianmeng and Justin, with Jianmeng becoming the main person responsible for hardware and Justin taking the group manager role. We do most of tasks allocations during the daily meetings, as then we know the amount of time that each person can spend on SDP during that day. Also, even though we got assigned specific roles, we were still putting a lot of effort to keep up to date on each others progress, mainly because this allowed us to allocate more people to help a subteam which gets a more demanding task.
	\section{Progress tracking}
	\subsection{Trello}
		As we needed to be sure that progress was achieved in a timely manner regardless of any milestones encountered, we decided to use a professional tool to help us manage progress tracking. Thus, we chose Trello as once we had to precisely split tasks, Trello seemed to provide a convenient and easy way of doing that. We set up a board where we had a list with the tasks we gave ourselves for that particular week. This decision  buttresses our perspective of handling an agile process with one week long iterations. 
		
		Every card created on Trello underlines the due date each task is supposed to be finished by. We tried to give ourselves realistic deadlines, even though sometimes they needed readjusting. Trello also has integration for automatically creating Gantt charts and we utilized this to better know how long each task is taking us to complete.This feature is essential as it showed us which areas needed more of our attention.
		
		Another way Trello proved useful is how it enables the creation of to-do lists for each task. Thus, splitting tasks further into smaller subtasks eased the entire development process as we always knew the next mini-goal we had and that managed to keep us on track. In addition to this, we used task comments section to note any problems we were encountering.
		
		\emph{GRAPH HERE}\newline
		\emph{GRAPH HERE}
		
		Also, we have allocated team members to each task on Trello in order to easily track who is working on what. Multiple people working on the code at the same time necessitated the use of a version control system too. Therefore, we had to integrate Git into the project. As some members of the teams were not used to working with it, we even had a quick tutorial about it that was held by Justin.
	\subsection{Git}
		\subsubsection{Using forks}
			At first every one of us forked the main repository and made changes to code in our own forks, as updating the main repository by making pull requests was the most secure approach.
		\subsubsection{Using separate branches}
			After using the forks approach for some time, we noticed that this approach is better suited for tasks that are less interconnected and it better suits less experienced Git users. We solved this issue by making a switch to simpler system, using only one repository and multiple branches on it. We set up a branch for every contributor and kept the master branch up.
			
	\section{Plans and Milestones}
		\subsection{Milestones}
			The main milestones that we encountered and will encounter in the future, are underlined in the following list:
			\begin{enumerate}
				\item January 22nd - Software basis sorted, robot moves
				\item January 24th - 3 holonomic wheel robot assembled, basic movement works
				\item January 28th - Robot is capable of grabbing and kicking
				\item February 1st - Friendly match 1
				\item February 15th - Friendly match 2
				\item February 17th - Process report submission
				\item March 1st - Friendly match 3
				\item March 22nd - Friendly match 4
				\item March 29th - Rehearsing of presentations
				\item April 5th - User guide submission
				\item April 7th - Final match day
				\item April 19th - Technical report submission
			\end{enumerate}
			However, even though we strictly followed our schedule, there were several extra roadblocks that we had to deal with:
			\begin{itemize}
				\item Adding sensor for ball detection in the grabber is a more difficult task than expected. The sonar which was given by electronic technician performed not as expected - it could detect only the objects which are further away from it, and thus is not suitable for our application. The laser sensor would be a better option though, but due to our technician being ill and delivery times taking more than a week, we cannot expect the sensor to arrive before second match. Thus we are trying colour sensor as a temporary solution.
				\item Failure of making the robot face the ball - we found out that the delay of data transfer, processing and sending Arduino commands is one of the main reasons why we struggle to make robot face the ball and face the gate. To solve this problem, we allocated more time for code optimisation.
			\end{itemize}
		\subsection{Work plan}
			Our work plan is split into weeks, because of our adherence to the Agile management process. Each major task is split into subtasks, which then are assigned to one or more members of the team, depending on difficulty and the qualification of the people working on it.
			\emph{Achieved - until the second friendly match}
			During the first week our team was working to get first two goals done. We focused on setting up a basis on which both hardware and software groups could work further.
			\begin{itemize}
				\item The first goal for a hardware team was to create a four holonomic wheels robot which could be used for testing the code belonging to teams from previous years. In order to achieve this task, the team had to be:
				\begin{enumerate}
					\item Going to the hardware workshop to get introduction on structurally rigid robot building (20 January)
					\item Building a copy of Fred (a 2016 Holonomic Drive Example System) (20-21 January)
					\item Setting up RF stick for connection between Arduino in the bot and DICE machine. (20-21 January)
					\item Loading Arduino C++ code to the robot, which enables the controls of motors and propeller. (21 January)
				\end{enumerate}
				\item The software team was working on basic implementation of vision, strategy and communication which entailed the following subtasks: 
				\begin{enumerate}
					\item Going to the software workshop to get introduction on all vital parts of the software system. (20 January)
					\item Researching on both holonomic drive example system (Java) and 2 wheel drive example system (Python). (20-21 January)
					\item Selecting the preferable system (holonomic drive example) based on overall performance, quality of vision system and quite easy to understand strategy. 
					\item Implementing and testing the Fred system on our first iteration of Fred robot. (22-24 January)
				\end{enumerate}
			\end{itemize}
			As we finished these two goals during the first half of the week - earlier than we expected,  we started thinking more about our performance at the first match. We identified the main area which required major improvement - the propeller, which was a really bad implementation of a kicker, and set new tasks for software and hardware teams.
			\begin{itemize}
				\item Hardware team had to come up with robot design which could accommodate a kicker. It meant that the whole robot had to go through major changes and we went through several iterations of design.
				\begin{enumerate}
					\item Trying to implement kicker into a robot which uses four wheels (23 January)
					\item Implementing a three holonomic wheels system due to lack of space for kicker and questionable rigidity of the design shown above. (Software development depended greatly on decisions made by hardware the team, thus in one day a basic three wheeled robot without kicker was built for movement strategy testing.) (23 January)
					\item Designing rigid and space saving robot implementation with a kicker. (23-24 January)
				\end{enumerate}
				\item Software team had to adjust the project code to work with the changes made by the hardware team.
				\begin{enumerate}
					\item Write and test code made for the kicker on the test-robot which has basic kicker and four wheels. (23-25 January)
					\item Edit and test movement code to work with the three wheeled robot design. (24-27 January)
					\item Implement the changes to the code which would allow the three wheeled robot go towards the ball and kick it with new kicker. (27 January)
				\end{enumerate}
			\end{itemize}
			The second part of our plan was done on 27th of January, at around the middle of the second week. There was still room for improvement for our first match,thus we established a plan for the time period until the first match:
			\begin{itemize}
				\item Hardware team had to:
				\begin{enumerate}
					\item Add a grabber, so that the robot would be able to catch the ball and turn towards the gate. (28 January)
					\item Make the kicker more precise, as the rules require our robot to kick straight. (28 January)
					\item Add a rigid holders for batteries and boards on the robot. - 28 January
					Fix the green plate on top of the robot without blocking the access to arduino and other boards connectors. (28 January)
				\end{enumerate}
				\item Software team had these tasks:
				\begin{enumerate}
					\item Add support for the grabbing action. (28-29 January)
					\item Find and implement the way for the robot to detect that the ball is in the grabber. (28-29 January)
					\item Make the robot face the goal when it catches the ball and then kick it. (29-31 January)
				\end{enumerate}
			\end{itemize}
			After winning the match during first friendly tournament we decided to concentrate our work on fixing the flaws which we noticed during the fights. The plan spanning from February 2nd all the way to 15th of February covered:
			\begin{itemize}
				\item Hardware team:
				\begin{enumerate}
					\item Prevent robot from flipping over when it drives with one wheel onto the incline near the side of the wall.
					\item Add sensor which could detect the ball in the grabber.
					\item Switch the wiring between motors and arduino, so that the software team could get readings on rotation of motors.
				\end{enumerate}
				\item Software team:
				\begin{enumerate}
					\item Increase the accuracy of the robot when it is trying to face the goal.
					\item Fix the case when the robot is driving with ball in grabber.
					\item Fix the vision system, which loses robot a lot more often in Room 1, mainly because of higher amount of glare and more uneven lighting conditions.
				\end{enumerate}
			\end{itemize}
			\emph{Desired - after the second friendly match}
			As we are following the Agile methodology, our work plan becomes more comprehensive on a weekly basis. However, we know the key points that have to exist in our work plan for the following weeks:
			\begin{itemize}
				\item Fixing the problems which we were unable to solve before the second match
				\item Starting a lot more thorough conversation with Group 1
				\begin{enumerate}
					\item Organising a team meeting, with both mentors involved.
					\item Discussing the options on joining strategy and/or vision.
					\item Deciding on roles of the robots (Defender/Attacker).
				\end{enumerate}
				\item We also need to do such tasks as:
				\begin{enumerate}
					\item Finalising the process report
					\item Schedule the work for the sprint week
				\end{enumerate}
			\end{itemize}
	\section{Risk Assessment and Contingency Planning}
		Inadvertently, we encountered several unexpected issues in the development process. As we were aware of these occurrences, we decided that a mandatory part of each meeting we have is discussing current issues and approaches to handle them. Some of these problems would often recur and, as a good practice, we decided to always have prepared solutions for them. Thus, we made the following risk assessment table:
		\begin{center}
			\begin{tabular}{|p{7cm}|p{7cm}|} 
				\hline
				Problem & Solution \\ [0.5ex] 
				\hline\hline
				Room conditions change constantly, vision breaks & Remember to save previous vision calibrations for different conditions, create new ones\\ 
				\hline
				Rooms are sometimes full, cannot test & Reevaluate due dates, do not leave things for last moment before matches\\
				\hline
				Code does not run when switching branches & Set up a guide on how to run it properly i.e. add the required libraries etc. \\
				\hline
				Certain task takes longer than expected & Reevaluate due dates, allocate more people to it, look at new approaches\\
				\hline
				A new feature breaks the code, makes the robot worse overall & Pull last working version from GitHub\\
				\hline
				Team members are unavailable & Reevaluate due dates, contact mentor\\
				\hline
				Robot batteries are dead & Make sure a pair is always charged, ask other teams if required\\
				\hline
				Robot keeps breaking in action & Reinforce robot structure\\
				\hline
				Cannot understand a team members' code & Everyone should properly add comments to their own pieces of code\\
				\hline
				Members with keys from lockers are absent & Leave a key at Forest Hill\\ [1ex] 
				\hline
			\end{tabular}
		\end{center}
	\section{Conclusion}
		Our group had and will continue to have a well organized structural organisation that can be noticed by our easy and continuous communication between group members, a flexible and comprehensive work plan, a profound awareness for the existence of manifold milestones, and an astute preparation when exposed to risks.	
\end{document}